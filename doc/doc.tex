% Options for packages loaded elsewhere
\PassOptionsToPackage{unicode}{hyperref}
\PassOptionsToPackage{hyphens}{url}
%
\documentclass[
]{article}
\usepackage{amsmath,amssymb}
\usepackage{iftex}
\ifPDFTeX
  \usepackage[T1]{fontenc}
  \usepackage[utf8]{inputenc}
  \usepackage{textcomp} % provide euro and other symbols
\else % if luatex or xetex
  \usepackage{unicode-math} % this also loads fontspec
  \defaultfontfeatures{Scale=MatchLowercase}
  \defaultfontfeatures[\rmfamily]{Ligatures=TeX,Scale=1}
\fi
\usepackage{lmodern}
\ifPDFTeX\else
  % xetex/luatex font selection
\fi
% Use upquote if available, for straight quotes in verbatim environments
\IfFileExists{upquote.sty}{\usepackage{upquote}}{}
\IfFileExists{microtype.sty}{% use microtype if available
  \usepackage[]{microtype}
  \UseMicrotypeSet[protrusion]{basicmath} % disable protrusion for tt fonts
}{}
\makeatletter
\@ifundefined{KOMAClassName}{% if non-KOMA class
  \IfFileExists{parskip.sty}{%
    \usepackage{parskip}
  }{% else
    \setlength{\parindent}{0pt}
    \setlength{\parskip}{6pt plus 2pt minus 1pt}}
}{% if KOMA class
  \KOMAoptions{parskip=half}}
\makeatother
\usepackage{xcolor}
\usepackage{color}
\usepackage{fancyvrb}
\newcommand{\VerbBar}{|}
\newcommand{\VERB}{\Verb[commandchars=\\\{\}]}
\DefineVerbatimEnvironment{Highlighting}{Verbatim}{commandchars=\\\{\}}
% Add ',fontsize=\small' for more characters per line
\newenvironment{Shaded}{}{}
\newcommand{\AlertTok}[1]{\textcolor[rgb]{1.00,0.00,0.00}{\textbf{#1}}}
\newcommand{\AnnotationTok}[1]{\textcolor[rgb]{0.38,0.63,0.69}{\textbf{\textit{#1}}}}
\newcommand{\AttributeTok}[1]{\textcolor[rgb]{0.49,0.56,0.16}{#1}}
\newcommand{\BaseNTok}[1]{\textcolor[rgb]{0.25,0.63,0.44}{#1}}
\newcommand{\BuiltInTok}[1]{\textcolor[rgb]{0.00,0.50,0.00}{#1}}
\newcommand{\CharTok}[1]{\textcolor[rgb]{0.25,0.44,0.63}{#1}}
\newcommand{\CommentTok}[1]{\textcolor[rgb]{0.38,0.63,0.69}{\textit{#1}}}
\newcommand{\CommentVarTok}[1]{\textcolor[rgb]{0.38,0.63,0.69}{\textbf{\textit{#1}}}}
\newcommand{\ConstantTok}[1]{\textcolor[rgb]{0.53,0.00,0.00}{#1}}
\newcommand{\ControlFlowTok}[1]{\textcolor[rgb]{0.00,0.44,0.13}{\textbf{#1}}}
\newcommand{\DataTypeTok}[1]{\textcolor[rgb]{0.56,0.13,0.00}{#1}}
\newcommand{\DecValTok}[1]{\textcolor[rgb]{0.25,0.63,0.44}{#1}}
\newcommand{\DocumentationTok}[1]{\textcolor[rgb]{0.73,0.13,0.13}{\textit{#1}}}
\newcommand{\ErrorTok}[1]{\textcolor[rgb]{1.00,0.00,0.00}{\textbf{#1}}}
\newcommand{\ExtensionTok}[1]{#1}
\newcommand{\FloatTok}[1]{\textcolor[rgb]{0.25,0.63,0.44}{#1}}
\newcommand{\FunctionTok}[1]{\textcolor[rgb]{0.02,0.16,0.49}{#1}}
\newcommand{\ImportTok}[1]{\textcolor[rgb]{0.00,0.50,0.00}{\textbf{#1}}}
\newcommand{\InformationTok}[1]{\textcolor[rgb]{0.38,0.63,0.69}{\textbf{\textit{#1}}}}
\newcommand{\KeywordTok}[1]{\textcolor[rgb]{0.00,0.44,0.13}{\textbf{#1}}}
\newcommand{\NormalTok}[1]{#1}
\newcommand{\OperatorTok}[1]{\textcolor[rgb]{0.40,0.40,0.40}{#1}}
\newcommand{\OtherTok}[1]{\textcolor[rgb]{0.00,0.44,0.13}{#1}}
\newcommand{\PreprocessorTok}[1]{\textcolor[rgb]{0.74,0.48,0.00}{#1}}
\newcommand{\RegionMarkerTok}[1]{#1}
\newcommand{\SpecialCharTok}[1]{\textcolor[rgb]{0.25,0.44,0.63}{#1}}
\newcommand{\SpecialStringTok}[1]{\textcolor[rgb]{0.73,0.40,0.53}{#1}}
\newcommand{\StringTok}[1]{\textcolor[rgb]{0.25,0.44,0.63}{#1}}
\newcommand{\VariableTok}[1]{\textcolor[rgb]{0.10,0.09,0.49}{#1}}
\newcommand{\VerbatimStringTok}[1]{\textcolor[rgb]{0.25,0.44,0.63}{#1}}
\newcommand{\WarningTok}[1]{\textcolor[rgb]{0.38,0.63,0.69}{\textbf{\textit{#1}}}}
\usepackage{graphicx}
\makeatletter
\def\maxwidth{\ifdim\Gin@nat@width>\linewidth\linewidth\else\Gin@nat@width\fi}
\def\maxheight{\ifdim\Gin@nat@height>\textheight\textheight\else\Gin@nat@height\fi}
\makeatother
% Scale images if necessary, so that they will not overflow the page
% margins by default, and it is still possible to overwrite the defaults
% using explicit options in \includegraphics[width, height, ...]{}
\setkeys{Gin}{width=\maxwidth,height=\maxheight,keepaspectratio}
% Set default figure placement to htbp
\makeatletter
\def\fps@figure{htbp}
\makeatother
\setlength{\emergencystretch}{3em} % prevent overfull lines
\providecommand{\tightlist}{%
  \setlength{\itemsep}{0pt}\setlength{\parskip}{0pt}}
\setcounter{secnumdepth}{-\maxdimen} % remove section numbering
\ifLuaTeX
  \usepackage{selnolig}  % disable illegal ligatures
\fi
\IfFileExists{bookmark.sty}{\usepackage{bookmark}}{\usepackage{hyperref}}
\IfFileExists{xurl.sty}{\usepackage{xurl}}{} % add URL line breaks if available
\urlstyle{same}
\hypersetup{
  hidelinks,
  pdfcreator={LaTeX via pandoc}}

\author{}
\date{}

\begin{document}

\section{基于LlamaIndex/LangChain构建的RAG-ChatGLM大模型知识库}\label{ux57faux4e8ellamaindexlangchainux6784ux5efaux7684rag-chatglmux5927ux6a21ux578bux77e5ux8bc6ux5e93}

\subsection{检索增强生成
(RAG)}\label{ux68c0ux7d22ux589eux5f3aux751fux6210-rag}

LLM
接受过大量数据的培训,但他们没有接受过\textbf{个人定制化}数据的训练。检索增强生成
(RAG) 通过将您的数据添加到LLM
已经有权访问的数据中来解决这个问题。您将在本文档中经常看到对 RAG
的引用。

在 RAG
中,您的数据已加载并准备好用于查询或``索引''。用户查询作用于索引,索引将数据过滤到最相关的上下文。然后,此上下文和您的查询连同提示一起转到LLM
,LLM会提供响应。

即使您正在构建的是聊天机器人或代理,您也会想了解将数据导入应用程序的 RAG
技术。

\includegraphics{C:/Users/14388/Downloads/rag-llm/assets/basic_rag.png}

\subsection{RAG 中的阶段}\label{rag-ux4e2dux7684ux9636ux6bb5}

RAG
中有五个关键阶段,这将成为您构建的任何大型应用程序的一部分。这些都是:

\begin{itemize}
\item
  \textbf{加载}:这是指将数据从其所在位置(无论是文本文件、PDF、其他网站、数据库还是
  API)获取到您的管道中。\href{https://llamahub.ai/}{LlamaHub}提供数百种连接器可供选择。
\item
  \textbf{索引}:这意味着创建一个允许查询数据的数据结构。对于LLM
  来说,这几乎总是意味着创建\texttt{vector\ embeddings}数据含义的数字表示,以及许多其他元数据策略,以便轻松准确地找到上下文相关的数据。
\item
  \textbf{存储}:一旦数据被索引,您几乎总是希望存储索引以及其他元数据,以避免重新索引。
\item
  \textbf{查询}:对于任何给定的索引策略,您可以通过多种方式利用 LLM 和
  LlamaIndex 数据结构进行查询,包括子查询、多步查询和混合策略。
\item
  \textbf{评估}:任何管道中的关键步骤是检查它相对于其他策略的有效性,或者何时进行更改。评估提供客观衡量您对查询的答复的准确性、忠实度和速度的程度。
\end{itemize}

\includegraphics{C:/Users/14388/Downloads/rag-llm/assets/stages.png}

\subsection{每个步骤中的重要概念}\label{ux6bcfux4e2aux6b65ux9aa4ux4e2dux7684ux91cdux8981ux6982ux5ff5}

您还会遇到一些术语,它们指的是每个阶段中的步骤。

\subsubsection{装载阶段}\label{ux88c5ux8f7dux9636ux6bb5}

\textbf{节点和文档}:A\texttt{Document}是任何数据源的容器 - 例如
PDF、API 输出或从数据库检索数据。 A\texttt{Node}是 LlamaIndex
中数据的原子单位,表示源的``块''\texttt{Document}。节点具有将它们与它们所在的文档以及其他节点相关联的元数据。

\textbf{连接器}:数据连接器(通常称为
a\texttt{Reader})将来自不同数据源和数据格式的数据摄取到\texttt{Documents}和\texttt{Nodes}。

\subsubsection{索引阶段}\label{ux7d22ux5f15ux9636ux6bb5}

\textbf{索引}:获取数据后,LlamaIndex
将帮助您将数据索引到易于检索的结构中。这通常涉及生成\texttt{vector\ embeddings}存储在称为\texttt{vector\ store}.索引还可以存储有关数据的各种元数据。

\textbf{嵌入}LLM 生成称为
的数据的数字表示\texttt{embeddings}。在过滤数据的相关性时,LlamaIndex
会将查询转换为嵌入,并且您的向量存储将查找在数字上与查询的嵌入相似的数据。

\subsubsection{查询阶段}\label{ux67e5ux8be2ux9636ux6bb5}

\textbf{检索器}:检索器定义在给定查询时如何从索引有效检索相关上下文。您的检索策略对于检索数据的相关性和检索效率至关重要。

\textbf{路由器}:路由器确定将使用哪个检索器从知识库中检索相关上下文。更具体地说,该类\texttt{RouterRetriever}
负责选择一个或多个候选检索器来执行查询。他们使用选择器根据每个候选人的元数据和查询来选择最佳选项。

\textbf{节点后处理器}:节点后处理器接收一组检索到的节点并对它们应用转换、过滤或重新排序逻辑。

\textbf{响应合成器}:响应合成器使用用户查询和给定的一组检索到的文本块从
LLM 生成响应。

\subsubsection{组合}\label{ux7ec4ux5408}

有数据支持的 LLM 应用程序有无数的用例,但它们可以大致分为三类:

\textbf{查询引擎}:查询引擎是一个端到端管道,允许您对数据提出问题。它接受自然语言查询,并返回响应,以及检索并传递给LLM
的参考上下文。

\textbf{聊天引擎}:聊天引擎是一个端到端的管道,用于与您的数据进行对话(多次来回而不是单个问答)。

\textbf{代理}:代理是由LLM
提供支持的自动化决策者,通过一组工具与世界进行交互。代理可以采取任意数量的步骤来完成给定的任务,动态地决定最佳的行动方案,而不是遵循预先确定的步骤。这赋予它额外的灵活性来处理更复杂的任务。

\subsection{核心技术分析}\label{ux6838ux5fc3ux6280ux672fux5206ux6790}

\subsubsection{LlamaIndex pipeline}\label{llamaindex-pipeline}

\textbf{一、载入环节}

\begin{enumerate}
\def\labelenumi{\arabic{enumi}.}
\item
  在载入过程中的"document"不是通用的理解的文档,它更是一种``容器''的代指,它可以是API、CSV、PDF、数据库记录等多种形式。在llamaIndex里有一个LlamaHub的组件,作用就是充当多种数据源的连接器,这个组件库也是开源的,现在已经有了不少connector还在持续更新中。
\end{enumerate}

\begin{enumerate}
\def\labelenumi{\arabic{enumi}.}
\item
  document里的主要属性有两个,一个是用来描述document的"metadata"(元数据),一个是用来描述document与其他document和node关系的relationships。
\item
  "node"也不是老IT术语里的``节点''的意思,而是指document中的切分出的一块,我觉得它的同义词应该是``shard''(分片)。
\end{enumerate}

调用llamaindex载入数据源成为document的基础代码是:

\begin{Shaded}
\begin{Highlighting}[]
\NormalTok{from llama\_index.core import SimpleDirectoryReader
}
\NormalTok{documents = SimpleDirectoryReader("./包含查询文档的文件夹名称").load\_data()}
\end{Highlighting}
\end{Shaded}

\textbf{二、建立索引}

通过llama\_index.core的另一个组件VectorStoreIndex来实现,从名字就可以看到,Index是按向量的方式进行存储的。

\begin{Shaded}
\begin{Highlighting}[]
\NormalTok{from llama\_index.core import VectorStoreIndex
}
\NormalTok{index = VectorStoreIndex.from\_documents(documents)}
\end{Highlighting}
\end{Shaded}

\textbf{三、查询环节}

\begin{enumerate}
\def\labelenumi{\arabic{enumi}.}
\item
  llamaindex是通过一个叫``query\_engine"的组件来进行查询的,通过在索引上建立的一个或多个retriever来实现。用户可以在提示中输入自然语言查询。普通查询的代码为:
\end{enumerate}

\begin{Shaded}
\begin{Highlighting}[]
\NormalTok{query\_engine = index.as\_query\_engine()}
\NormalTok{response = query\_engine.query("要搜索的内容")}
\end{Highlighting}
\end{Shaded}

\begin{enumerate}
\def\labelenumi{\arabic{enumi}.}
\item
  query engine里最重要的组成部分就是retriever,
  retriever可以被进一步配置完成更加精细化的查询设置。在llamaindex的简化案例中,没有动用retriever,应该采用的就是query
  engine的默认设置。
\item
  查询里还有个离不开的部分是把LLM大模型结合进来,负责LLM的部分在llamaIndex的query
  engine里叫做``Synthesizer``(合成器),它登场的时机在node从retriever中被提取出,以及进行完处理转换后。默认情况下不调整synthesizer也能得到LLM参与的结果,但是进行配置后可以得到更佳的结果。
\end{enumerate}

llama的文档中给了个配置synthesizer的例子,这个response\_mode里的参数``compact"可以把数据块提前压缩,减少对大模型token的调用量,起到节约开销的效果。

\begin{Shaded}
\begin{Highlighting}[]
\ImportTok{from}\NormalTok{ llama\_index.core.data\_structs }\ImportTok{import}\NormalTok{ Node
}
\ImportTok{from}\NormalTok{ llama\_index.core.schema }\ImportTok{import}\NormalTok{ NodeWithScore
}
\ImportTok{from}\NormalTok{ llama\_index.core }\ImportTok{import}\NormalTok{ get\_response\_synthesizer
}


\NormalTok{response\_synthesizer }\OperatorTok{=}\NormalTok{ get\_response\_synthesizer(response\_mode}\OperatorTok{=}\StringTok{"compact"}\NormalTok{)
}


\NormalTok{response }\OperatorTok{=}\NormalTok{ response\_synthesizer.synthesize(
}
    \StringTok{"query text"}\NormalTok{, nodes}\OperatorTok{=}\NormalTok{[NodeWithScore(node}\OperatorTok{=}\NormalTok{Node(text}\OperatorTok{=}\StringTok{"需搜索的问题"}\NormalTok{), score}\OperatorTok{=}\FloatTok{1.0}\NormalTok{)]
}
\NormalTok{)}
\end{Highlighting}
\end{Shaded}

我用了llamaIndex教程文档里的示例文章提问,查询问题为``What are the two
things the author worked on?''采用普通查询模式时,得到的结果为:

\begin{Shaded}
\begin{Highlighting}[]
\NormalTok{The author worked on writing and programming. }
\end{Highlighting}
\end{Shaded}

在采用配置后的synthesizer后,得到的查询结果为:

\begin{Shaded}
\begin{Highlighting}[]
\NormalTok{The author worked on developing a new software application and conducting research on the impact of}
\end{Highlighting}
\end{Shaded}

\subsubsection{Advanced-RAG(高级检索增强生成)}\label{advanced-ragux9ad8ux7ea7ux68c0ux7d22ux589eux5f3aux751fux6210}

\includegraphics{C:/Users/14388/Downloads/rag-llm/assets/18z-QRadKewNmos0J_4TNAQ.png}

\paragraph{检索前优化}\label{ux68c0ux7d22ux524dux4f18ux5316}

检索前优化集中在数据索引优化和查询优化上。数据索引优化技术旨在以有助于提高检索效率的方式存储数据,例如
{[}1{]}:

\begin{enumerate}
\def\labelenumi{\arabic{enumi}.}
\item
  滑动窗口使用片段之间的重叠,是最简单的技术之一。
\item
  提高数据粒度应用数据清洗技术,例如删除无关信息、确认事实准确性、更新过时信息等。
\item
  添加元数据,如日期、目的或章节,用于过滤目的。
\item
  优化索引结构涉及不同的策略来索引数据,例如调整片段大小或使用多索引策略。本文将实现的一项技术是句子窗口检索,它将单个句子嵌入到检索中,并在推断时用更大的文本窗口替换它们。
\end{enumerate}

\includegraphics{C:/Users/14388/Downloads/rag-llm/assets/1pbU5KBqYWhx0GOeNRMiPoA.png}

此外,检索前技术不仅限于数据索引,还可以涉及推理时的技术,如查询路由、查询重写和查询扩展。

\paragraph{检索优化}\label{ux68c0ux7d22ux4f18ux5316}

检索阶段的目标是确定最相关的上下文。通常,检索基于向量搜索,它计算查询与索引数据之间的语义相似性。因此,大多数检索优化技术都围绕嵌入模型展开
{[}1{]}:

\begin{enumerate}
\def\labelenumi{\arabic{enumi}.}
\item
  微调嵌入模型,将嵌入模型定制为特定领域的上下文,特别是对于术语不断演化或罕见的领域。例如,BAAI/bge-small-en是一个高性能的嵌入模型,可以进行微调(请参阅微调指南)。
\item
  动态嵌入根据单词的上下文进行调整,而静态嵌入则为每个单词使用单一向量。例如,OpenAI的embeddings-ada-02是一个复杂的动态嵌入模型,可以捕获上下文理解。{[}1{]}
\end{enumerate}

除了向量搜索之外,还有其他检索技术,例如混合搜索,通常是指将向量搜索与基于关键字的搜索相结合的概念。如果您的检索需要精确的关键字匹配,则此检索技术非常有益。

\paragraph{检索后优化}\label{ux68c0ux7d22ux540eux4f18ux5316}

\includegraphics{C:/Users/14388/Downloads/rag-llm/assets/1owudWLuoXhqeLjStCsnDiw.png}

对检索到的上下文进行额外处理可以帮助解决一些问题,例如超出上下文窗口限制或引入噪声,从而阻碍对关键信息的关注。在RAG调查中总结的检索后优化技术包括:

\begin{enumerate}
\def\labelenumi{\arabic{enumi}.}
\item
  提示压缩:通过删除无关内容并突出重要上下文,减少整体提示长度。
\item
  重新排序:使用机器学习模型重新计算检索到的上下文的相关性得分。
\end{enumerate}

\paragraph{实现高级RAG代码部分}\label{ux5b9eux73b0ux9ad8ux7ea7ragux4ee3ux7801ux90e8ux5206}

\begin{itemize}
\item
  检索前优化:句子窗口检索
\item
  检索优化:混合搜索
\item
  检索后优化:重新排序
\end{itemize}

\textbf{索引优化示例:句子窗口检索}

对于句子窗口检索技术,您需要进行两个调整:首先,您必须调整如何存储和后处理您的数据。我们将使用
SentenceWindowNodeParser,而不是 SimpleNodeParser。

\begin{Shaded}
\begin{Highlighting}[]
\ImportTok{from}\NormalTok{ llama\_index.core.node\_parser }\ImportTok{import}\NormalTok{ SentenceWindowNodeParser
}


\NormalTok{node\_parser }\OperatorTok{=}\NormalTok{ SentenceWindowNodeParser.from\_defaults(
}
\NormalTok{    window\_size}\OperatorTok{=}\DecValTok{3}\NormalTok{,
}
\NormalTok{    window\_metadata\_key}\OperatorTok{=}\StringTok{"window"}\NormalTok{,
}
\NormalTok{    original\_text\_metadata\_key}\OperatorTok{=}\StringTok{"original\_text"}\NormalTok{,
}
\NormalTok{)}
\end{Highlighting}
\end{Shaded}

SentenceWindowNodeParser 做了两件事情:

\begin{itemize}
\item
  它将文档分成单个句子,这些句子将被嵌入。
\item
  对于每个句子,它创建一个上下文窗口。如果您指定 window\_size =
  3,则生成的窗口将为三个句子长,从嵌入句子的前一个句子开始,并跨越后一个句子。窗口将存储为元数据。
\end{itemize}

在检索过程中,返回与查询最接近的句子。检索后,您需要通过定义
MetadataReplacementPostProcessor
并在节点后处理器列表中使用它来用元数据替换句子。

\begin{Shaded}
\begin{Highlighting}[]
\ImportTok{from}\NormalTok{ llama\_index.core.postprocessor }\ImportTok{import}\NormalTok{ MetadataReplacementPostProcessor
}


\CommentTok{\# The target key defaults to \textasciigrave{}window\textasciigrave{} to match the node\_parser\textquotesingle{}s default
}
\NormalTok{postproc }\OperatorTok{=}\NormalTok{ MetadataReplacementPostProcessor(
}
\NormalTok{    target\_metadata\_key}\OperatorTok{=}\StringTok{"window"}

\NormalTok{)
}
\NormalTok{...
}
\NormalTok{query\_engine }\OperatorTok{=}\NormalTok{ index.as\_query\_engine( 
}
\NormalTok{    node\_postprocessors }\OperatorTok{=}\NormalTok{ [postproc],
}
\NormalTok{)}
\end{Highlighting}
\end{Shaded}

\textbf{检索优化示例:混合搜索}

在LlamaIndex中实现混合搜索与两个参数更改相同,如果底层向量数据库支持混合搜索查询的话。alpha
参数指定向量搜索和基于关键字的搜索之间的加权,其中 alpha = 0

表示基于关键字的搜索,alpha = 1 表示纯向量搜索。

\begin{Shaded}
\begin{Highlighting}[]
\NormalTok{query\_engine }\OperatorTok{=}\NormalTok{ index.as\_query\_engine(
}
\NormalTok{    ...,
}
\NormalTok{    vector\_store\_query\_mode}\OperatorTok{=}\StringTok{"hybrid"}\NormalTok{, 
}
\NormalTok{    alpha}\OperatorTok{=}\FloatTok{0.5}\NormalTok{,
}
\NormalTok{    ...
}
\NormalTok{)}
\end{Highlighting}
\end{Shaded}

\textbf{检索后优化示例:重新排序}

将 reranker 添加到您的高级RAG管道中只需要三个简单的步骤:

首先,定义一个重新排序模型。在这里,我们使用来自Hugging Face的
BAAI/bge-reranker-base。
在查询引擎中,将重新排序模型添加到节点后处理器列表中。 在查询引擎中增加
similarity\_top\_k 以检索更多的上下文段落,在重新排序后可以将其减少到
top\_n。

\begin{Shaded}
\begin{Highlighting}[]
\CommentTok{\# !pip install torch sentence{-}transformers
}
\ImportTok{from}\NormalTok{ llama\_index.core.postprocessor }\ImportTok{import}\NormalTok{ SentenceTransformerRerank
}


\CommentTok{\# Define reranker model
}
\NormalTok{rerank }\OperatorTok{=}\NormalTok{ SentenceTransformerRerank(
}
\NormalTok{    top\_n }\OperatorTok{=} \DecValTok{2}\NormalTok{, 
}
\NormalTok{    model }\OperatorTok{=} \StringTok{"BAAI/bge{-}reranker{-}base"}

\NormalTok{)
}
\NormalTok{...
}
\CommentTok{\# Add reranker to query engine
}
\NormalTok{query\_engine }\OperatorTok{=}\NormalTok{ index.as\_query\_engine(
}
\NormalTok{        similarity\_top\_k }\OperatorTok{=} \DecValTok{6}\NormalTok{,
}
\NormalTok{        ...,
}
\NormalTok{                node\_postprocessors }\OperatorTok{=}\NormalTok{ [rerank],
}
\NormalTok{        ...,
}
\NormalTok{)}
\end{Highlighting}
\end{Shaded}

\subsection{参考}\label{ux53c2ux8003}

\begin{quote}
{[}1{]} Gao, Y., Xiong, Y., Gao, X., Jia, K., Pan, J., Bi, Y., \ldots{}
\& Wang, H. (2023). Retrieval-augmented generation for large language
models: A survey. arXiv preprint arXiv:2312.10997.

{[}2{]}
\url{https://towardsdatascience.com/advanced-retrieval-augmented-generation-from-theory-to-llamaindex-implementation-4de1464a9930}
\end{quote}

\end{document}
